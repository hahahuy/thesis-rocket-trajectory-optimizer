\newpage

\begin{center}
    \fontsize{14}{16}\selectfont
    \textbf{ABSTRACT}
\end{center}
\vspace{-0.3cm}
\doublespacing
\justifying
This thesis investigates physics-informed neural network (PINN) surrogates for six-degree-of-freedom (6-DOF) launch vehicle ascent dynamics, with the goal of approximating high-fidelity trajectory solutions at substantially reduced computational cost. Starting from a rigid-body 6-DOF dynamics model and a direct optimal control formulation, a dataset of numerically optimal ascent trajectories is generated using collocation-based methods. These trajectories serve as supervision for training structured neural surrogates that predict full state evolution from time and mission context parameters.

The primary contribution of this work is the design and evaluation of a structured PINN architecture, referred to as Direction AN, which combines Fourier time embeddings, a context encoder for physical parameters, a shared residual backbone, and specialized output branches for translational motion, rotational dynamics, and mass depletion. Physical consistency is enforced through a physics residual loss constructed from finite-difference approximations of the continuous-time equations of motion, together with additional structural constraints such as quaternion normalization, mass monotonicity, and boundary consistency. A phased training strategy is employed to balance data fidelity, physics regularization, and smoothing terms, improving stability during learning.
    
The resulting surrogate accurately reconstructs position, velocity, attitude, and mass trajectories over the launch phase (0–30 s) across a range of sounding-rocket configurations, producing smooth and physically plausible ascent profiles consistent with a high-fidelity truth integrator. While the surrogate is not used directly to optimize control inputs in this study, its differentiable structure and physics-aware design make it suitable for future integration into gradient-based trajectory optimization and guidance frameworks. Overall, the results demonstrate that carefully regularized PINN surrogates can serve as effective and scalable approximations of complex rocket dynamics, providing a foundation for fast analysis and optimization in ascent trajectory design.




