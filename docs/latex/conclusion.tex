\chapter{Conclusion and recommendations}
\section{Conclusions}
This thesis investigated whether a structured physics-informed neural 
network can function as a surrogate model for rocket ascent dynamics 
and be embedded within a trajectory optimization context. The study was 
motivated by the computational cost and limited flexibility of classical 
optimal control solvers when repeated trajectory evaluations or gradient 
information are required.

The results support the conclusion that physics-informed neural networks 
constitute a viable surrogate modeling approach for rocket ascent 
dynamics when the governing equations are well understood and training 
data are available. By embedding the equations of motion directly into 
the learning objective, the proposed model is able to reconstruct 
full-state trajectories that are globally consistent with the reference 
dynamics without relying on explicit numerical integration during 
inference. This confirms the central hypothesis that physics guidance can 
significantly improve the reliability of neural surrogates compared to 
purely data-driven models.

In line with prior work on physics-informed learning, the study shows that 
state variables governed primarily by integral effects are captured more 
robustly than derivative-dominated quantities. This observation is 
consistent with findings reported by Raissi et al. and Karniadakis et al., 
where PINNs were shown to approximate smooth solution manifolds 
effectively while remaining sensitive to higher-order dynamics. The 
present work extends these insights to a six-degree-of-freedom rocket 
ascent problem with coupled translational, rotational, and mass dynamics.

At the same time, the thesis demonstrates that soft enforcement of physical 
constraints is insufficient to guarantee strict compliance with 
non-negotiable physical laws such as mass monotonicity and symmetry 
preservation. While the model learns the overall structure of propellant 
consumption and orientation dynamics, local violations remain possible when 
constraint satisfaction competes with data fidelity. This aligns with 
recent theoretical analyses of PINN optimization behavior, which emphasize 
the trade-off between loss terms and the absence of hard guarantees in 
penalty-based formulations.

Overall, the findings indicate that physics-informed neural networks are well 
suited as analysis and optimization surrogates, rather than direct 
replacements for high-fidelity simulators in safety-critical applications. 
Their value lies in enabling differentiable, computationally efficient 
approximations that preserve the dominant physical behavior of complex dynamical 
systems.

\section{Implications for Trajectory Optimization and Scientific Machine Learning}

From a trajectory optimization perspective, the proposed surrogate 
framework offers a promising pathway toward reducing reliance on repeated 
numerical integration in gradient-based optimization loops. The 
differentiable nature of the model enables direct sensitivity analysis 
with respect to initial conditions or control parameters, which is 
difficult to achieve efficiently with traditional solvers.

Within the broader context of scientific machine learning, this work 
reinforces the view that architectural structure and loss design are as 
important as network capacity. The observed behavior of the model 
highlights the need for careful alignment between physical constraints, 
training objectives, and the intended downstream use of the surrogate. 
The study thus contributes practical evidence to ongoing discussions in 
the PINN literature regarding robustness, constraint enforcement, and 
long-horizon prediction.

\section{Recommendations for Future Work}

Based on the conclusions drawn from this study, several recommendations 
are proposed for future research and practical development.

First, future models should incorporate hard or semi-hard constraint 
enforcement mechanisms for critical physical laws, particularly mass 
monotonicity and symmetry conditions. Projection methods, constrained 
output parameterizations, or hybrid solver-network approaches may provide 
stronger physical guarantees than penalty-based losses alone.

Second, phase-aware or adaptive training strategies are recommended to 
address regions of rapid dynamic change, such as engine cutoff or thrust 
transitions. Dynamically adjusting loss weights or introducing curriculum 
learning could improve derivative accuracy without sacrificing global 
trajectory consistency.

Third, extending the framework to variable time horizons and non-nominal 
flight conditions, including wind disturbances or parameter uncertainty, 
would significantly enhance its applicability to real-world scenarios. 
Such extensions would also provide a more rigorous test of generalization 
beyond the training regime.

Finally, integration with real flight data or higher-fidelity simulation 
environments is recommended to evaluate robustness and practical relevance. 
Combining physics-informed learning with data assimilation techniques may 
offer a path toward hybrid models capable of bridging the gap between simulation 
and operational deployment.

\section{Final Remarks}
This thesis demonstrates that physics-informed neural networks can serve as 
meaningful surrogate models for rocket ascent dynamics when designed and 
evaluated carefully. While the approach does not eliminate the need for 
classical simulation tools, it provides a complementary methodology that
supports fast evaluation, differentiability, and physical consistency.

The insights gained from this work underscore both the potential and the 
limitations of PINN-based surrogates, offering a foundation for future 
research aimed at integrating machine learning more deeply into aerospace 
trajectory analysis and optimization workflows.