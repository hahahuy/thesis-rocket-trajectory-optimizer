\chapter{DISCUSSION AND IMPLICATIONS}


\section{Overall Performance of the Physics-Informed Model}

The results demonstrate that the proposed Physics-Informed Neural Network is capable 
of reconstructing the global structure of rocket trajectories when evaluated on unseen 
test cases. The model successfully captures the dominant trends of vertical position, 
velocity, and mass evolution without explicit access to control inputs during inference. 
This confirms that embedding physical laws directly into the learning objective provides 
sufficient inductive bias for trajectory reconstruction in high-dimensional dynamical 
systems.

The close agreement observed in vertical position throughout the ascent phase indicates 
that the network effectively learns the integral effects of acceleration and thrust over 
time. This aligns with established findings in physics-informed learning, where state 
variables governed by smooth integral dynamics tend to be reconstructed more accurately 
than their time derivatives.

At the same time, deviations observed toward the end of the trajectory suggest that 
accumulated error remains a challenge, particularly in long-horizon predictions. This 
behavior is consistent with prior PINN studies applied to nonlinear ordinary differential 
equations, where error propagation becomes more pronounced as the temporal distance from the 
initial condition increases.

\section{Interpretation of Velocity and Higher-Order Dynamics}
The vertical velocity results reveal larger discrepancies compared to position, especially 
during later stages of flight. This outcome is expected, as velocity depends directly on 
first-order derivatives of the network output with respect to time. In physics-informed 
models, derivative-based quantities are more sensitive to approximation error and numerical 
stiffness.

The divergence observed in the predicted velocity near the terminal time suggests that the 
network prioritizes global trajectory consistency over local derivative accuracy. This 
trade-off reflects a known characteristic of PINNs, where minimizing a combined data and 
physics loss does not guarantee uniform accuracy across all state derivatives.

From a physical perspective, velocity dynamics are influenced by rapidly changing forces, 
including thrust cutoff and mass variation. Capturing these transitions accurately requires 
either increased temporal resolution or stronger enforcement of the governing equations 
during these phases. The present implementation demonstrates reasonable performance but 
indicates that velocity prediction remains a limiting factor for precise trajectory reconstruction.

\section{Mass Evolution and Constraints Enforcement}

The mass evolution results show that the network learns the general trend of propellant 
consumption during the burn phase and the constant-mass regime after cutoff. However, 
small oscillations around the terminal mass and multiple monotonicity violations are observed 
in the predicted trajectory.

These violations indicate that while the mass conservation law is included as a physics 
residual, it is not enforced as a hard constraint. As a result, the model allows small 
physically inconsistent fluctuations if they reduce the overall loss. This behavior highlights 
an important limitation of soft constraint enforcement in PINNs.

In practical aerospace applications, mass monotonicity is a non-negotiable physical constraint. 
The current results suggest that additional structural constraints or post-processing steps 
are necessary if the model is to be deployed in safety-critical settings. Nonetheless, the 
ability of the network to approximate the mass profile without explicit integration 
demonstrates the flexibility of the physics-informed approach.

\section{Horizontal Drift and Symmetry Breaking}

The horizontal position results reveal small but persistent deviations from the reference 
trajectory, which remains identically zero. This indicates that the network does not 
strictly preserve symmetry in unconstrained dimensions.

Such drift is a known phenomenon in neural differential equation models, particularly 
when the governing equations admit trivial solutions. In the absence of explicit penalties 
or symmetry-enforcing constraints, the optimizer may converge to nearby solutions that 
satisfy the loss function but violate problem-specific invariances.

From a modeling standpoint, this observation suggests that additional regularization terms 
or coordinate-specific constraints may be required when zero-motion states are physically 
expected. Alternatively, reparameterizing the model to exclude irrelevant degrees of freedom 
could reduce spurious predictions and improve robustness.

\section{Quaternion Normalization and Attitude Representation}
The quaternion norm deviation results indicate that the predicted orientation remains 
extremely close to unit norm throughout the trajectory. This confirms that the normalization 
strategy used during training is effective at preserving valid attitude representations.

Maintaining quaternion normalization is critical for rotational dynamics, as even small 
deviations can lead to non-physical rotations when propagated over time. The present results 
show that the chosen formulation successfully mitigates this risk without introducing 
numerical instability.

This outcome aligns with existing literature on learning-based attitude modeling, where 
explicit normalization or penalty-based approaches are necessary to ensure physically 
meaningful orientation states.

\section{Physics Residual Behavior and Model Consistency}

The physics residual analysis provides insight into how well the learned solution satisfies 
the governing equations. The residual magnitudes remain bounded throughout the trajectory 
and decrease toward the end of the time horizon.

This trend suggests that the network gradually converges toward a solution that is 
increasingly consistent with the physical model. However, the non-uniform residual profile 
indicates that constraint satisfaction varies across different flight phases.

Such behavior is typical in PINNs applied to nonlinear systems with time-varying dynamics. 
Residual dominance may shift between terms depending on the operating regime, highlighting 
the importance of adaptive loss weighting or curriculum-based training strategies.

\section{Practical Implications and Applicability}

The results indicate that physics-informed neural networks are a viable surrogate modeling 
approach for rocket trajectory reconstruction. The ability to generate full state trajectories 
without explicit numerical integration suggests potential applications in rapid simulation, 
design space exploration, and embedded onboard prediction systems.

However, the observed constraint violations and derivative inaccuracies indicate that the 
current model is better suited for analysis and approximation rather than direct deployment 
in control loops. For industrial or mission-critical applications, additional safeguards and 
validation steps would be required.

The approach remains attractive for scenarios where computational efficiency and differentiability 
are prioritized, such as optimization, inverse problems, or sensitivity analysis.

\section{Limitations of the Present Study}

Several limitations should be acknowledged. First, the model is evaluated on simulated data 
generated from the same underlying physics used in training. Generalization to real flight 
data with unmodeled disturbances remains untested.

Second, constraint enforcement relies entirely on soft penalties, which allows physically 
inconsistent solutions when trade-offs favor loss minimization. This is particularly evident 
in mass monotonicity and horizontal drift.

Third, the fixed time horizon and uniform sampling may limit the model’s ability to handle 
variable-duration flights or abrupt dynamic transitions.

These limitations do not invalidate the results but highlight areas where further development 
is necessary.

\section{Recommendations for Future Improvements}

Future work should explore hybrid constraint enforcement strategies that combine physics-informed 
losses with hard constraints or projection-based corrections. Enforcing mass monotonicity and 
symmetry directly at the architectural level may significantly improve physical consistency.

Adaptive loss weighting and phase-aware training could improve performance during critical 
dynamic transitions, such as engine cutoff. Additionally, extending the framework to handle 
variable time horizons and stochastic disturbances would enhance practical applicability.

Incorporating real-world flight data or higher-fidelity aerodynamic models would further 
test the robustness and generalization capability of the proposed approach.

