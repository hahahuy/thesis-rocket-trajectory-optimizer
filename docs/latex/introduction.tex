% Introduction chapter for the rocket trajectory optimisation project
% This file is intended to be included from a main thesis document.

\section{Introduction}

Modern launch vehicles are critically constrained by trajectory design. Small
improvements in ascent guidance can translate into significant gains in payload
capacity, mission robustness, and operational safety. At the same time, the
governing dynamics of a rocket in atmospheric flight are highly nonlinear and
strongly coupled: translational and rotational motion interact through
aerodynamic forces, thrust vectoring, and mass depletion, all under tight
structural and environmental constraints.

This project develops a high-fidelity six-degree-of-freedom (6-DOF) dynamics
model for a launch vehicle, together with a direct optimal control formulation
of the ascent trajectory problem and data-driven surrogate models based on
physics-informed neural networks (PINNs). The resulting framework can generate
physically consistent optimal trajectories, enforce realistic path constraints
such as limits on dynamic pressure and load factor, and provide differentiable
approximations of the dynamics for downstream optimisation and analysis.

The present document introduces the underlying physical model, the numerical
methods used to solve the optimal control problem, and the current state and
planned evolution of the overall methodology.

\section{Physical Model}
\label{sec:physical-model}

\subsection{Notation}
\label{subsec:notation}

\begin{table}[h]
\centering
\caption{Notation and typical parameter ranges}
\label{tab:notation}
\begin{tabular}{llll}
\toprule
Symbol & Description & Units & Typical Range \\
\midrule
\(\mathbf{r}_\mathrm{i}\) & Position (inertial) & m & -- \\
\(\mathbf{v}_\mathrm{i}\) & Velocity (inertial) & m/s & 0--500 \\
\(\mathbf{q}\) & Quaternion (attitude) & -- & unit norm \\
\(\boldsymbol{\omega}_\mathrm{b}\) & Angular velocity (body) & rad/s & -- \\
\(m\) & Mass & kg & 35--65 \\
\(T\) & Thrust magnitude & N & 0--5000 \\
\(\theta_\mathrm{g}, \phi_\mathrm{g}\) & Gimbal angles & rad & \(\pm 0.1745\) \\
\(\delta\) & Control surface deflection & rad & \(\pm 0.1745\) \\
\(m_0\) & Initial mass & kg & 45--65 \\
\(I_\mathrm{sp}\) & Specific impulse & s & 220--280 \\
\(C_\mathrm{D}\) & Drag coefficient & -- & 0.25--0.45 \\
\(C_{\mathrm{L}\alpha}\) & Lift-curve slope & rad\(^{-1}\) & 2.5--4.5 \\
\(C_{\mathrm{m}\alpha}\) & Pitch-moment coefficient & rad\(^{-1}\) & -1.2--(-0.4) \\
\(T_\mathrm{max}\) & Maximum thrust & N & 3000--5000 \\
\(q_\mathrm{dyn,max}\) & Max dynamic pressure & Pa & 40,000 \\
\(n_\mathrm{max}\) & Max load factor & g & 5.0 \\
\bottomrule
\end{tabular}
\end{table}

Table~\ref{tab:notation} summarizes the key notation used throughout this
document, including state and control variables, physical parameters, and
their typical ranges or units.

\subsection{State, Control, and Reference Frames}

The rocket is modelled as a rigid body with full 6-DOF dynamics. The state
vector \(x \in \mathbb{R}^{14}\) is
\[
  x =
  \begin{bmatrix}
    \mathbf{r}_\mathrm{i}^\top &
    \mathbf{v}_\mathrm{i}^\top &
    \mathbf{q}^\top &
    \boldsymbol{\omega}_\mathrm{b}^\top &
    m
  \end{bmatrix}^\top ,
\]
where
\begin{itemize}
  \item \(\mathbf{r}_\mathrm{i} \in \mathbb{R}^3\) is the position of the
        vehicle in an Earth-centred inertial frame,
  \item \(\mathbf{v}_\mathrm{i} \in \mathbb{R}^3\) is the inertial velocity,
  \item \(\mathbf{q} = [q_0, q_1, q_2, q_3]^\top\) is a unit quaternion
        representing the attitude (body-to-inertial rotation),
  \item \(\boldsymbol{\omega}_\mathrm{b} \in \mathbb{R}^3\) is the angular
        velocity expressed in the body frame, and
  \item \(m \in \mathbb{R}\) is the mass of the vehicle.
\end{itemize}

The control vector \(u \in \mathbb{R}^4\) is
\[
  u =
  \begin{bmatrix}
    T &
    \theta_\mathrm{g} &
    \phi_\mathrm{g} &
    \delta
  \end{bmatrix}^\top ,
\]
where
\begin{itemize}
  \item \(T\) is the commanded thrust magnitude,
  \item \(\theta_\mathrm{g}\) and \(\phi_\mathrm{g}\) are the thrust gimbal
        pitch and yaw angles, and
  \item \(\delta\) is a representative control surface deflection.
\end{itemize}

The body-to-inertial rotation matrix \(R_{\mathrm{b}\to\mathrm{i}}(\mathbf{q})\)
is obtained from the quaternion \(\mathbf{q}\). Its transpose
\(R_{\mathrm{i}\to\mathrm{b}} = R_{\mathrm{b}\to\mathrm{i}}^\top\) transforms
vectors from the inertial frame to the body frame.

\paragraph{Physical scale and typical values.}
The model is parameterised for small sounding rockets with initial masses
ranging from 45 to 65 kg, maximum thrusts of 3000--5000 N, and specific
impulses of 220--280 s. Typical trajectories reach apogees of 30--42 km
over 30-second flight durations, with maximum velocities of approximately
300--500 m/s. The aerodynamic coefficients vary within realistic ranges:
drag coefficient \(C_\mathrm{D} \in [0.25, 0.45]\), lift-curve slope
\(C_{\mathrm{L}\alpha} \in [2.5, 4.5]\) rad\(^{-1}\), and pitch-moment
coefficient \(C_{\mathrm{m}\alpha} \in [-1.2, -0.4]\) rad\(^{-1}\). Path
constraints enforce maximum dynamic pressure of 40,000 Pa and maximum load
factor of 5 g, consistent with structural limits for small launch vehicles.
Typical propellant consumption ranges from 10 to 30 kg over the flight
duration (corresponding to initial masses of 45--65 kg and a dry mass of
35 kg), with gimbal angles remaining small (less than 10°) for
near-vertical ascent trajectories.

\paragraph{Control representation across the pipeline.}
Different parts of the implementation use different control representations
for numerical and storage efficiency. The CasADi-based optimal control solver
and PyTorch PINN dynamics use the gimbal-angle parameterisation
\(u = [T, \theta_\mathrm{g}, \phi_\mathrm{g}, \delta]^\top\) described above.
The C++ truth integrator uses a unit-vector representation
\(\tilde{u} = [T, u_{\mathrm{T},x}, u_{\mathrm{T},y}, u_{\mathrm{T},z}, \delta]^\top\),
where the thrust direction components are related to the gimbal angles by
\[
  u_{\mathrm{T},x} = \cos\theta_\mathrm{g}\cos\phi_\mathrm{g}, \qquad
  u_{\mathrm{T},y} = \sin\phi_\mathrm{g}, \qquad
  u_{\mathrm{T},z} = \sin\theta_\mathrm{g}\cos\phi_\mathrm{g}.
\]
Processed datasets store control as \([T, u_{\mathrm{T},x}, u_{\mathrm{T},y}, u_{\mathrm{T},z}]\)
(4-D format, omitting \(\delta\) which is typically zero). The control surface
deflection \(\delta\) is optimised in the CasADi formulation and included in the
C++ dynamics when non-zero, but is not used in PINN training, where a zero-control
assumption is employed (see Section~\ref{sec:methodology}).

\subsection{Forces, Moments, and Mass Flow}

The translational motion is governed by Newton's second law,
\[
  \dot{\mathbf{r}}_\mathrm{i} = \mathbf{v}_\mathrm{i}, \qquad
  \dot{\mathbf{v}}_\mathrm{i} =
  \frac{1}{m}\,\mathbf{F}_\mathrm{i}(\cdot) + \mathbf{g}_\mathrm{i}(\mathbf{r}_\mathrm{i}),
\]
where \(\mathbf{F}_\mathrm{i}\) is the total non-gravitational force in the
inertial frame and \(\mathbf{g}_\mathrm{i}\) is the gravitational acceleration.
In the high-fidelity C++ ``truth'' model, gravity is modelled as an
inverse-square law,
\[
  \mathbf{g}_\mathrm{i}(\mathbf{r}_\mathrm{i}) =
  - g_0 \left(\frac{R_\mathrm{E}}{\|\mathbf{r}_\mathrm{i}\|}\right)^2
  \frac{\mathbf{r}_\mathrm{i}}{\|\mathbf{r}_\mathrm{i}\|},
\]
with Earth radius \(R_\mathrm{E}\) and nominal surface gravity \(g_0\). In the
CasADi- and PyTorch-based models used for optimal control and PINN training,
a constant gravitational field \(\mathbf{g}_\mathrm{i} = [0, 0, -g_0]^\top\) is
employed for simplicity and numerical robustness.

Aerodynamic and thrust forces are naturally expressed in the body frame.
The relative wind velocity is
\[
  \mathbf{v}_\mathrm{rel,i} = \mathbf{v}_\mathrm{i} - \mathbf{v}_\mathrm{wind,i},
  \qquad
  \mathbf{v}_\mathrm{rel,b} =
  R_{\mathrm{i}\to\mathrm{b}}(\mathbf{q})\,\mathbf{v}_\mathrm{rel,i},
\]
where \(\mathbf{v}_\mathrm{wind,i}\) is the wind velocity in the inertial
frame. The C++ dynamics support a wind callback; in the current optimal control
and PINN implementations, wind is neglected and
\(\mathbf{v}_\mathrm{wind,i} = \mathbf{0}\).

The atmosphere is modelled by an exponential density profile,
\[
  \rho(h) = \rho_0 \exp\!\left(-\frac{h}{h_\mathrm{scale}}\right),
\]
where \(h\) is altitude above sea level, \(\rho_0\) is the sea-level density,
and \(h_\mathrm{scale}\) is a scale height. The dynamic pressure is
\[
  q_\mathrm{dyn} = \frac{1}{2}\,\rho(h)\,\|\mathbf{v}_\mathrm{rel,b}\|^2.
\]

Drag and lift forces in the body frame take the form
\[
  \mathbf{F}_\mathrm{D,b}
    = - q_\mathrm{dyn} S_\mathrm{ref} C_\mathrm{D}\,\hat{\mathbf{v}}_\mathrm{rel,b},
  \qquad
  \mathbf{F}_\mathrm{L,b}
    = q_\mathrm{dyn} S_\mathrm{ref} C_{\mathrm{L}\alpha}\,\alpha\,
      \hat{\mathbf{e}}_\mathrm{L},
\]
where \(S_\mathrm{ref}\) is a reference area, \(C_\mathrm{D}\) is the drag
coefficient, \(C_{\mathrm{L}\alpha}\) is a lift-curve slope, \(\alpha\) is the
angle of attack computed from \(\mathbf{v}_\mathrm{rel,b}\), and
\(\hat{\mathbf{e}}_\mathrm{L}\) is a unit lift direction approximately
perpendicular to the body \(x\)-axis and the relative velocity.

The thrust vector in the body frame is
\[
  \mathbf{u}_T =
  \begin{bmatrix}
    \cos\theta_\mathrm{g}\cos\phi_\mathrm{g} \\
    \sin\phi_\mathrm{g} \\
    \sin\theta_\mathrm{g}\cos\phi_\mathrm{g}
  \end{bmatrix},
  \qquad
  \mathbf{F}_\mathrm{T,b} = T\,\frac{\mathbf{u}_T}{\|\mathbf{u}_T\|},
\]
so that the total non-gravitational force in the body frame is
\[
  \mathbf{F}_\mathrm{b} =
  \mathbf{F}_\mathrm{T,b} +
  \mathbf{F}_\mathrm{D,b} +
  \mathbf{F}_\mathrm{L,b},
  \qquad
  \mathbf{F}_\mathrm{i} =
  R_{\mathrm{b}\to\mathrm{i}}(\mathbf{q})\,\mathbf{F}_\mathrm{b}.
\]

Rotational dynamics are expressed as
\[
  \dot{\mathbf{q}} =
    \tfrac{1}{2}\,\mathbf{q} \otimes
    \begin{bmatrix}0 \\ \boldsymbol{\omega}_\mathrm{b}\end{bmatrix},
  \qquad
  \dot{\boldsymbol{\omega}}_\mathrm{b}
    = \mathbf{I}_\mathrm{b}^{-1}
      \Bigl(
        \mathbf{M}_\mathrm{b} -
        \boldsymbol{\omega}_\mathrm{b} \times
        (\mathbf{I}_\mathrm{b}\boldsymbol{\omega}_\mathrm{b})
      \Bigr),
\]
where \(\mathbf{I}_\mathrm{b}\) is the inertia tensor in the body frame and
\(\mathbf{M}_\mathrm{b}\) is the sum of aerodynamic and thrust moments,
including gimbal and control-surface effects via pitch-moment and control
derivatives.

The mass dynamics follow the standard rocket equation,
\[
  \dot{m} = -\frac{T}{I_\mathrm{sp} g_0},
\]
with specific impulse \(I_\mathrm{sp}\) and surface gravity \(g_0\). A dry-mass
limit is enforced in the implementation to prevent unphysical mass depletion.

\subsection{Assumptions and Limitations}

The current model assumes a spherical, non-rotating Earth, neglects higher
gravitational harmonics and planetary rotation, and uses a single-vehicle rigid
body without staging or flexible modes. The atmosphere is represented by a
single-parameter exponential model; detailed thermodynamic and compositional
effects are neglected. Wind is configurable in the C++ truth integrator but is
set to zero in the optimisation and PINN configurations.

\paragraph{Motivation for simplifications.}
Each simplification is chosen to balance fidelity with computational
tractability while maintaining accuracy for the target application. The
spherical Earth assumption is acceptable for altitudes below 100 km (typical
trajectories reach 30--42 km), where gravitational variations are less than
1\%. Constant gravity in the optimization formulation reduces computational
cost while maintaining accuracy for short-duration flights; the error
introduced is negligible compared to other uncertainties. The exponential
atmosphere model captures over 90\% of the density variation relevant to
ascent trajectories; more complex models (e.g., US Standard Atmosphere) add
minimal value for the ascent phase while significantly increasing computational
complexity. The single-stage rigid-body assumption focuses the model on the
ascent phase, where staging events are not present; flexible modes are
negligible for small rockets with high structural stiffness (typical natural
frequencies exceed 100 Hz, well above the trajectory dynamics). Wind effects
are neglected in optimization to reduce parameter space dimensionality, but
the framework supports wind models for future extensions. These simplifications
yield a model that is rich enough to capture the key couplings between
translation, rotation, aerodynamics, and thrust, while remaining tractable for
large-scale optimal control and data generation.

\section{Numerical Solution and Optimal Control Formulation}
\label{sec:numerical-solution}

\subsection{Direct Optimal Control and Transcription}

The ascent trajectory design problem is posed as a continuous-time optimal
control problem (OCP) over a finite horizon \(t \in [0, t_f]\), with the
6-DOF dynamics from Section~\ref{sec:physical-model} as constraints. Typical
objectives include minimising fuel consumption or maximising delivered payload,
subject to path and terminal constraints on altitude, velocity, attitude, and
structural loads.

In this project the OCP is discretised by a direct collocation method. The
continuous state and control trajectories are approximated on a uniform grid of
nodes \(\{t_k\}_{k=0}^N\) with step size \(h = t_f/N\). At each node the state
\(x_k \approx x(t_k)\) and control \(u_k \approx u(t_k)\) become decision
variables in a finite-dimensional nonlinear programme (NLP). The dynamics are
enforced via Hermite--Simpson collocation, which provides third-order accuracy
by introducing a collocation point at the midpoint of each interval and
enforcing a defect constraint of the form
\[
  x_{k+1}
  = x_k + \frac{h}{6}
    \bigl(
      f(x_k, u_k) + 4 f(x_m, u_m) + f(x_{k+1}, u_{k+1})
    \bigr),
\]
where \(f(x,u)\) denotes the state derivative, \(x_m\) is a midpoint state
constructed from \((x_k,x_{k+1})\) and \((f(x_k,u_k), f(x_{k+1},u_{k+1}))\), and
\(u_m\) is the midpoint control. The resulting collocation defects are enforced
as equality constraints in the NLP.

\subsection{CasADi-Based Dynamics and IPOPT Solver}

The 6-DOF dynamics, including aerodynamics, thrust, and mass depletion, are
implemented symbolically in CasADi. This provides exact Jacobians and Hessians
to the NLP solver, which is crucial for robustness and performance in the
presence of stiff dynamics and tight path constraints. Special care is taken in
the implementation to avoid non-differentiabilities and division by small
quantities, for example by using smooth norm approximations and explicit
clamping of mass and thrust.

The discretised OCP is solved using IPOPT, a large-scale interior-point
nonlinear optimiser. State and control variables are scaled using reference
length, velocity, time, mass, and force scales to improve conditioning, and
linear solver backends (such as MUMPS or HSL variants) are selected
automatically depending on availability. The framework supports both fixed and
free final time, with appropriate bounds on \(t_f\) when treated as a decision
variable.

Path constraints on dynamic pressure \(q_\mathrm{dyn}\), load factor \(n\), and
mass are enforced directly in the NLP using auxiliary CasADi functions for
these quantities. Operational limits on gimbal angles, control-surface
deflections, and angle of attack are also encoded in the state and control
bounds. The solver returns optimal knot sequences
\(\{x_k^\star\}_{k=0}^N\), \(\{u_k^\star\}_{k=0}^{N-1}\), an optimal final time
\(t_f^\star\), and detailed convergence statistics.

\subsection{Truth Integration and PINN Dynamics}

In addition to the collocation-based OCP solver, the project includes a
high-fidelity C++ integrator for the same 6-DOF dynamics, which is used as a
``truth'' model for validation and data generation. This integrator supports
inverse-square gravity, exponential atmosphere, aerodynamic forces and
moments, wind callbacks, and detailed diagnostic outputs such as dynamic
pressure, load factor, and constraint violations. A Python wrapper interface is
planned to expose this integrator as a function on uniform time grids; at
present, the wrapper is defined but not yet fully implemented.

For data-driven approximation, a differentiable version of the dynamics is
implemented in PyTorch. This module mirrors the CasADi model, including the
state and control definitions, aerodynamic coefficients, and mass dynamics, but
uses tensor operations to enable automatic differentiation. It serves as the
physics backbone for physics-informed neural networks and hybrid PINN
architectures, which learn to reproduce or augment the 6-DOF dynamics from
simulated optimal trajectories.

\section{Methodology, Current Status, and Planned Extensions}
\label{sec:methodology}

\subsection{Overall Methodology}

The project methodology is organised around three tightly coupled components:
\begin{enumerate}
  \item a high-fidelity 6-DOF dynamics model (truth integrator),
  \item a CasADi- and IPOPT-based optimal control solver using direct
        collocation, and
  \item data-driven surrogate models based on PINNs and related architectures.
\end{enumerate}
The workflow proceeds as follows. First, the physical model is specified and
implemented consistently across C++, CasADi, and PyTorch, including a common
set of physical parameters and operational limits. Second, the direct
collocation solver is configured and tuned to produce dynamically feasible,
constraint-satisfying ascent trajectories across a range of mission scenarios.
Third, these optimal trajectories are used to construct datasets (stored in
HDF5 format) for training and validating PINN-based surrogate models of the
dynamics and, where appropriate, of the optimal control policy.

\subsection{Current State of Data and Models}

On the dynamics and optimisation side, the 6-DOF model is fully implemented in
C++ and in symbolic CasADi form, including aerodynamic forces and moments,
thrust vector control, and mass depletion. The direct collocation
transcription, Hermite--Simpson defects, state and control bounds, and path
constraints on dynamic pressure, load factor, and mass are implemented and
interfaced with IPOPT. Solver configuration includes robust scaling, automatic
linear-solver selection, and diagnostics for constraint violation and iteration
statistics.

The data pipeline produces a hierarchy of datasets. At the lowest level, raw
cases are stored as individual HDF5 files containing time grids, full 14-D
state trajectories, control histories, monitor signals (e.g.\ dynamic pressure
and load factor), and rich metadata including physical parameters, solver
statistics, and configuration hashes. These raw cases are then aggregated into
processed split files (train/validation/test) in nondimensional form. The
processed datasets contain approximately 120 training cases, 20 validation
cases, and 20 test cases. Each trajectory is discretised on a uniform time
grid with \(N = 1501\) points, corresponding to a 30-second flight duration
sampled at 50\,Hz. Each processed dataset exposes a time grid, a context
vector of dimension 7 encoding key physical and environmental parameters
(initial mass \(m_0\), specific impulse \(I_\mathrm{sp}\), drag coefficient
\(C_\mathrm{D}\), lift-curve slope \(C_{\mathrm{L}\alpha}\), pitch-moment
coefficient \(C_{\mathrm{m}\alpha}\), maximum thrust \(T_\mathrm{max}\), and
wind magnitude), and normalised state targets; an extended ``v2'' format
additionally provides time series of thrust magnitude and dynamic pressure
as explicit input features.

\paragraph{Nondimensionalization and scaling.}
All state and control variables are nondimensionalized using physics-aware
reference scales to improve numerical conditioning and enable scale-invariant
learning. The reference scales are chosen to yield quantities of order unity:
length scale \(L_\mathrm{ref} = 10,000\) m (characteristic altitude), velocity
scale \(V_\mathrm{ref} = 313\) m/s (approximately \(\sqrt{g_0 L_\mathrm{ref}}\)),
time scale \(T_\mathrm{ref} = 31.62\) s (approximately \(L_\mathrm{ref}/V_\mathrm{ref}\)),
mass scale \(M_\mathrm{ref} = 50\) kg (nominal wet mass), and force scale
\(F_\mathrm{ref} = 490\) N (approximately \(M_\mathrm{ref} g_0\)). This
nondimensionalization ensures that all state components are typically in the
range \([0.1, 10]\), which improves gradient flow during neural network
training and reduces numerical errors in the optimization solver. The scaling
also makes the model scale-invariant: trajectories for different-sized rockets
can be learned using the same architecture, with physical scale encoded in the
context vector.

On the modelling side, several families of PINN and hybrid architectures have
been implemented in PyTorch. Common building blocks include Fourier feature
embeddings of time, shallow and deep encoders for the context vector, and MLP
or Transformer-based temporal encoders. Sequence PINNs treat the entire time
grid as a sequence and apply Transformer encoders to predict the 14-D state at
all nodes. Hybrid latent-ODE models combine an encoder that infers a latent
initial condition \(z_0\) from early-time behaviour and context with a learned
latent dynamics model that evolves \(z(t)\) over time, followed by decoders or
dedicated branches that reconstruct translation, rotation (with explicit
quaternion normalisation or minimal parametrisations), and mass trajectories.
Several designs enforce structural properties such as monotonically decreasing
mass and unit-norm quaternions by construction, and all models can use the
differentiable 6-DOF dynamics module as a physics regulariser in the training
loss to promote physically consistent predictions.

\subsection{Direction AN Model and Training Objective}

Among the available architectures, the ``Direction AN'' model is a canonical
example of how the project combines learned representations with physics-based
regularisation. Its structure is organised into three stages (see
Figure~\ref{fig:direction_an_architecture}). A shared stem first embeds the
normalised time and context: time is expanded by Fourier features, while the
context vector is passed through a small encoder and then concatenated with the
time embedding. This combined input is processed by a residual multilayer
perceptron with skip connections and optional layer normalisation, yielding a
latent feature sequence \(z(t, c)\) of fixed dimension along the trajectory.

On top of this shared latent representation, three ``mission branches''
specialise to different parts of the 14-D state. A translation branch maps
\(z\) to position and velocity components \([\mathbf{r}_\mathrm{i}, 
\mathbf{v}_\mathrm{i}]\), a rotation branch maps \(z\) to quaternion and
angular velocity components \([\mathbf{q}, \boldsymbol{\omega}_\mathrm{b}]\)
with explicit quaternion renormalisation, and a mass branch produces the mass
trajectory \(m(t)\). The concatenation of these branch outputs forms the
predicted state \(x_\theta(t)\). A variant of the architecture (AN1) replaces
the plain stem input with a richer block that also ingests time series of
thrust magnitude and dynamic pressure, fusing \((t, c, T_\mathrm{mag}(t),
q_\mathrm{dyn}(t))\) into the shared latent representation.

\paragraph{Architecture design rationale.}
The Direction AN architecture is designed to leverage the physical structure
of the 6-DOF dynamics while enabling efficient learning. The shared stem
captures common temporal and physical dependencies that affect all state
components (e.g., altitude-dependent atmospheric effects, time-varying mass).
Separate mission branches decouple the learning of translation, rotation, and
mass dynamics, which have different physical characteristics: translation
follows Newton's laws with aerodynamic coupling, rotation involves quaternion
kinematics and moment dynamics, and mass follows a simple depletion law. This
decoupling allows each branch to specialize while sharing common features
through the stem. The explicit quaternion renormalisation in the rotation
branch ensures valid rotations (unit quaternions) by construction, avoiding
the need for post-processing or soft constraints. The physics residual layer
directly enforces the 6-DOF dynamics equations, ensuring that predictions
satisfy the underlying physics even when data is sparse or noisy. This
hybrid approach combines the expressiveness of neural networks with the
rigor of physics-based constraints.

\begin{figure}[h]
\centering
\includegraphics[width=0.8\textwidth]{figures/direction_an_architecture.pdf}
\caption{Architecture of the Direction AN model. The shared stem processes
time and context inputs to produce a latent representation \(z(t, c)\). Three
specialized branches (translation, rotation, mass) map this latent space to
the corresponding state components. The physics residual layer computes
dynamics residuals to enforce physical consistency.}
\label{fig:direction_an_architecture}
\end{figure}

The training objective for Direction AN follows a physics-informed pattern.
Given a reference trajectory \(x^\star(t_k)\) on a uniform grid
\(\{t_k\}_{k=0}^{N}\), the total loss is a weighted sum
\[
  \begin{split}
  L
  &= \lambda_\mathrm{data} L_\mathrm{data}
  + \lambda_\mathrm{phys} L_\mathrm{phys}
  + \lambda_\mathrm{bc} L_\mathrm{bc}
  + \lambda_\mathrm{quat} L_\mathrm{quat}
  + \lambda_\mathrm{mass} L_\mathrm{mass}
  + \lambda_\mathrm{mass,res} L_\mathrm{mass,res} \\
  &\quad + \lambda_\mathrm{vz,res} L_\mathrm{vz,res}
  + \lambda_\mathrm{vxy,res} L_\mathrm{vxy,res}
  + \lambda_\mathrm{smooth,z} L_\mathrm{smooth,z}
  + \lambda_\mathrm{smooth,vz} L_\mathrm{smooth,vz} \\
  &\quad + \lambda_\mathrm{pos,vel} L_\mathrm{pos,vel}
  + \lambda_\mathrm{smooth,pos} L_\mathrm{smooth,pos}
  + \lambda_\mathrm{zero,vxy} L_\mathrm{zero,vxy}
  + \lambda_\mathrm{zero,axy} L_\mathrm{zero,axy} \\
  &\quad + \lambda_\mathrm{hacc} L_\mathrm{hacc}
  + \lambda_\mathrm{xy,zero} L_\mathrm{xy,zero}
  + \cdots ,
  \end{split}
\]
where the additional terms beyond the core physics loss represent soft
constraints and smoothing regularisers. The data term \(L_\mathrm{data}\) is
a component-weighted mean-squared error between predicted and true states,
with separate group weights for translation, rotation, and mass so that
altitude, vertical velocity, and attitude can be emphasised. The physics term
\(L_\mathrm{phys}\) penalises violations of the continuous dynamics by
comparing a finite-difference estimate of the time derivative
\(\dot{x}_\theta(t_k)\) to the right-hand side
\(f(x_\theta(t_k), u_k; p)\) of the 6-DOF model implemented in PyTorch,
\[
  L_\mathrm{phys}
  = \frac{1}{N} \sum_{k=0}^{N}
    \bigl\|
      \dot{x}_\theta(t_k) - f(x_\theta(t_k), u_k; p)
    \bigr\|^2 .
\]
The control \(u_k\) in the physics loss is set to zero
(\(u_k = [0, 0, 0, 0]^\top\)), reflecting the intentional design choice that
PINNs learn trajectories without explicit control input; the control is
implicitly encoded in the state trajectory itself. This zero-control assumption
is consistent with the processed dataset format, which does not include control
trajectories, and enables the model to learn the mapping from initial
conditions and physical parameters directly to state evolution.

A boundary term \(L_\mathrm{bc}\) enforces agreement between predicted and true
initial conditions, while regularisers such as the quaternion norm penalty
\(L_\mathrm{quat}\) and a mass-flow consistency term \(L_\mathrm{mass}\)
encourage unit quaternions and non-increasing mass along the trajectory.

\paragraph{Loss function term interpretations.}
The additional soft losses serve specific physical and numerical purposes:
\begin{itemize}
  \item \(L_\mathrm{mass,res}\): Ensures mass decreases monotonically,
        preventing unphysical mass gain that could arise from data fitting
        alone.
  \item \(L_\mathrm{vz,res}\): Encourages realistic vertical velocity
        evolution, ensuring the model captures the acceleration--deceleration
        profile of ascent trajectories.
  \item \(L_\mathrm{vxy,res}\): Suppresses spurious horizontal velocity
        components, as sounding rockets should follow near-vertical trajectories.
  \item \(L_\mathrm{smooth,z}\), \(L_\mathrm{smooth,vz}\): Temporal smoothing
        terms that prevent oscillatory predictions in altitude and vertical
        velocity, which can arise from noisy data or insufficient physics
        constraints.
  \item \(L_\mathrm{smooth,pos}\): Smooths position predictions to ensure
        physically plausible trajectories without sudden jumps.
  \item \(L_\mathrm{pos,vel}\): Enforces kinematic consistency between
        position and velocity (i.e., \(\dot{\mathbf{r}} = \mathbf{v}\)).
  \item \(L_\mathrm{zero,vxy}\), \(L_\mathrm{zero,axy}\): Suppress horizontal
        velocities and accelerations during near-vertical flight, as lateral
        motion should be minimal for sounding rockets.
  \item \(L_\mathrm{hacc}\), \(L_\mathrm{xy,zero}\): Additional constraints on
        horizontal acceleration and position to ensure vertical ascent
        behavior.
\end{itemize}
These regularization terms work together to ensure that the learned model
produces physically plausible trajectories that respect the expected behavior
of sounding rocket ascent, even when training data is limited or contains
outliers.

Training employs a phased schedule where loss weights are adjusted over time.
In the initial phase (approximately the first 55\% of training epochs), the
model focuses on fitting the data and coarse physics constraints, with lower
weights on higher-order regularisation terms. In the second phase, these
regularisation weights are gradually increased using a cosine ramp schedule,
allowing the model to refine its predictions while maintaining physical
consistency. This phased approach helps prevent the model from overfitting to
the regularisation terms early in training while ensuring that physical
constraints are properly enforced in the final model. Together, these elements
yield a training objective that balances data fidelity with dynamical
consistency and structural robustness.

\paragraph{RMSE calculation.}
Model performance is evaluated using root mean square error (RMSE) computed on
the test set. For a trained Direction AN model, predictions
\(\hat{x}_\theta(t_k)\) are generated for all test trajectories on a uniform time
grid \(\{t_k\}_{k=0}^{N-1}\) spanning the trajectory duration (typically 30
seconds). The total RMSE is computed as
\[
  \mathrm{RMSE}_\mathrm{total} = \sqrt{\frac{1}{B \cdot N \cdot D} \sum_{b=1}^{B}
  \sum_{k=0}^{N-1} \sum_{d=1}^{D} \left(\hat{x}_\theta^{(b)}(t_k)_d - 
  x^\star{(b)}(t_k)_d\right)^2},
\]
where \(B\) is the number of test trajectories (typically 20), \(N = 1501\) is the
number of time points in the discretized trajectory grid (corresponding to a
sampling rate of approximately 50 Hz), \(D = 14\) is the state dimension, and
\(x^\star{(b)}(t_k)\) denotes the ground truth state for trajectory \(b\) at time
\(t_k\). The summation over \(k\) from \(0\) to \(N-1\) averages the squared
errors across all time points in each trajectory, while the summation over \(b\)
averages across all test trajectories, and the summation over \(d\) averages
across all state components. Per-component RMSE is computed by averaging over
trajectories and time points for each state dimension \(d\), while aggregated
metrics are computed for translation (components 1--6: position and velocity),
rotation (components 7--13: quaternion and angular velocity), and mass
(component 14). All RMSE values are reported in nondimensional units, reflecting
the scaled state representation used during training. The current Direction AN
model achieves a total RMSE of approximately 0.197 on the test set, with
translation RMSE of 0.264, rotation RMSE of 0.133, and mass RMSE of 0.015.

\subsection{Planned Extensions}

Several extensions are planned to increase fidelity and robustness:
\begin{itemize}
  \item incorporating more detailed atmospheric and wind models in both the
        truth integrator and the OCP formulation;
  \item modelling staging events, thrust profiles, and mass properties that
        vary with propellant depletion and configuration changes;
  \item enriching the objective functions to include measures of robustness,
        controllability, or dispersion sensitivity;
  \item extending the PINN and latent-dynamics models to capture a wider range
        of operating conditions and to support uncertainty quantification.
\end{itemize}
Further work will also focus on integrating the C++ truth integrator more
tightly with the Python/ML stack, enabling closed-loop validation of learned
controllers and surrogates against the highest-fidelity model.