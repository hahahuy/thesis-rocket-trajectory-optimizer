% Introduction chapter for the thesis on physics-informed surrogate modelling
% for rocket ascent trajectory dynamics. This file is intended to be included
% from the main thesis document.

\chapter{Introduction}
\vspace{-1cm}
\doublespacing
\justifying
\section{Background and Motivation}

Trajectory optimization for rocket ascent is a classical and central problem in
aerospace engineering. The objective is to determine thrust and steering
profiles that guide a launch vehicle from liftoff to a desired terminal state
while satisfying nonlinear dynamics, path constraints, and structural or
operational limits. Traditionally, this problem has been addressed using
indirect methods derived from Pontryagin's Maximum Principle or direct
transcription approaches such as collocation and pseudospectral methods, which
convert the continuous-time optimal control problem into a large-scale
nonlinear programming problem~\cite{bryson1975,ross2004,betts2010}.

These classical methods are theoretically rigorous and widely used in practice,
but they rely on repeated numerical integration of nonlinear ordinary
differential equations and the solution of large constrained optimization
problems. As a result, they can be computationally expensive, sensitive to
initialization, and difficult to embed within higher-level design loops or
real-time decision-making frameworks. This limitation has motivated growing
interest in surrogate modelling techniques that can approximate system
dynamics while remaining computationally efficient and differentiable.

In parallel, advances in scientific machine learning have introduced
physics-informed neural networks (PINNs) as a framework for incorporating
physical laws directly into neural network training. Originally proposed by
Raissi et al.~\cite{raissi2019}, PINNs augment standard data-driven loss
functions with penalties enforcing governing differential equations, enabling
neural networks to learn solutions that are consistent with known physics.
Subsequent work has extended this idea to a wide range of forward and inverse
problems in physics and engineering, highlighting both the promise and the
challenges of the approach~\cite{karniadakis2021,pinnbook2022}.

Together, these developments suggest an opportunity to revisit rocket ascent
modelling from a new perspective, in which neural networks are not used purely
as black-box approximators, but as physics-guided surrogates that retain
computational efficiency, differentiability, and physical structure while
reducing reliance on repeated numerical integration.

\subsection{State of the Art}

Recent research in physics-informed machine learning has demonstrated the
potential of PINNs to model complex dynamical systems governed by ordinary and
partial differential equations. Comprehensive surveys have summarized the
strengths and limitations of the PINN framework, including challenges related
to optimization stiffness, loss balancing, and generalization
behaviour~\cite{karniadakis2021,pinnbook2022}. Analytical studies have further
examined PINN training dynamics and failure modes, emphasizing the importance
of architectural design and loss construction for practical
success~\cite{wang2022,mishra2022}.

In aerospace applications, surrogate modelling has traditionally relied on
response surfaces, reduced-order models, or purely data-driven neural
networks. While such models can be effective within narrow operating regimes,
they often struggle to extrapolate beyond the training domain and provide
limited physical interpretability. PINNs offer a middle ground by combining
data fidelity with physics-based regularization, making them particularly
attractive for modelling rocket ascent dynamics, where governing equations are
well understood but computationally expensive to evaluate repeatedly.

Despite this promise, the application of PINNs to constrained optimal control
problems remains relatively underexplored, especially for rocket ascent.
Existing studies predominantly focus on forward or inverse dynamics problems,
while fewer works investigate structured PINN surrogates as replacements for
high-fidelity dynamics models within trajectory optimization pipelines. This
gap is especially pronounced for six-degree-of-freedom (6-DOF) ascent
problems, which involve strong nonlinearities, coupling between translational
and rotational motion, and mass depletion effects.

\subsection{Rationale}

The central motivation of this thesis is to investigate whether a structured
physics-informed neural network can serve as a differentiable surrogate for
rocket ascent dynamics. Rather than repeatedly integrating nonlinear ordinary
differential equations, the surrogate aims to approximate the full state
evolution directly as a function of time and mission parameters, while
remaining consistent with the underlying physical laws.

Such surrogates are particularly appealing for ascent analysis and preliminary
design studies, where rapid evaluation of trajectory behaviour and sensitivity
to physical parameters is required. By shifting the computational burden to an
offline training phase, the learned model can potentially support future
optimization and guidance frameworks that require fast, smooth, and physically
consistent approximations of rocket dynamics.

\subsection{Problem Statement}

The problem addressed in this thesis is the development of a
physics-informed neural network surrogate capable of approximating the
three-dimensional ascent dynamics of a rocket during the launch phase. The
surrogate is trained on full-state trajectories generated from high-fidelity
numerical simulations of a 6-DOF rigid-body dynamics model and is designed to
reproduce the coupled translational, rotational, and mass-depletion behaviour
of rocket ascent.

While the broader objective is to support constrained trajectory optimization,
the focus of this study is on the accurate and physically consistent
approximation of ascent dynamics, rather than on the direct computation of
optimal control inputs using the surrogate.

\subsection{Objectives}

The specific objectives of this thesis are as follows:
\begin{enumerate}
  \item To formulate a three-dimensional rocket ascent model suitable for
        physics-informed learning.
  \item To design a structured PINN architecture incorporating Fourier time
        embeddings, context encoding, and residual multilayer perceptrons.
  \item To train the surrogate using full-state trajectory data generated from
        high-fidelity numerical simulations.
  \item To evaluate the accuracy, stability, and physical consistency of the
        learned surrogate during the launch phase.
  \item To assess the suitability of the surrogate as a building block for
        future integration into constrained trajectory optimization frameworks.
\end{enumerate}

\subsection{Scope and Limitations}

The scope of this work is limited to three-dimensional rocket ascent dynamics
without wind disturbances. The surrogate is trained on trajectory data covering
the initial 30~seconds of flight, corresponding to the launch phase. As a
result, conclusions regarding full-ascent modelling, closed-loop control
optimization, and generalization beyond the training regime are outside the
scope of this thesis.

In particular, the study does not demonstrate closed-loop optimization of
control inputs using the learned surrogate. These limitations are acknowledged
and discussed as directions for future work.

\subsection{Research Framework}

The research framework follows a multi-level structure. At the data generation
level, high-fidelity numerical simulations and optimal control solvers are used
to produce dynamically feasible ascent trajectories. At the learning level,
these trajectories are used to train a physics-informed neural network
surrogate that approximates state evolution over time. At the application
level, the trained surrogate is evaluated as a candidate replacement for
numerical integration in future optimization and guidance workflows.

\subsection{Structure of the Thesis}

The remainder of this thesis is organized as follows. Chapter~2 reviews the
theoretical background and related work in physics-informed neural networks and
optimal control. Chapter~3 presents the rocket dynamics model and problem
formulation. Chapter~4 describes the PINN architecture, training methodology,
and surrogate evaluation. Chapter~5 reports the numerical results. Chapter~6
discusses the findings and limitations of the study. Chapter~7 concludes the
thesis and outlines directions for future research.




