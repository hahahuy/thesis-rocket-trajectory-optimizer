\chapter{Results}

\section{Evaluation Setup}

This chapter evaluates the performance of the proposed Physics-Informed Neural 
Network using a held-out test set that was not used during training. The 
objective is to assess trajectory reconstruction accuracy, quantitative error 
characteristics, and compliance with physical constraints.

The test dataset consists of multiple simulated rocket trajectories generated 
using a high-fidelity ordinary differential equation solver. Each trajectory 
spans a fixed time horizon of 30 seconds and includes translational states, 
rotational states represented by quaternions, and vehicle mass evolution. 
All trajectories are sampled at a uniform temporal resolution consistent with 
the training setup.

Prediction accuracy is evaluated by comparing the PINN outputs against the reference 
solver outputs at corresponding time instances. Absolute error is computed 
pointwise in time for representative trajectories, while root-mean-square 
error is used to quantify performance across the entire test set. All reported 
RMSE values are computed per trajectory and then aggregated across the test 
dataset.

No control inputs or reference trajectories are provided to the network during 
inference. The model predicts the full state evolution solely from the initial 
condition and time input.

\section{Trajectory Reconstruction Results}

This section presents a qualitative comparison between the PINN-predicted 
trajectory and the reference trajectory for a single representative test 
case. The purpose is to visually assess how well the model reconstructs the 
time evolution of different state components.

\subsection{Vertical Motion}

\begin{figure}[htbp]
    \centering
    \includegraphics[width=14.5cm]{figures/evaluation_report/figures/figure_3_1_vertical_position.png}
    \caption{shows the vertical position as a function of time. The PINN 
    prediction closely follows the reference solution over the full ascent phase, 
    with a visible but small deviation near the end of the time horizon.}
    \label{fig:vertical_motion}
\end{figure}

\begin{figure}[htbp]
    \centering
    \includegraphics[width=14.5cm]{figures/evaluation_report/figures/figure_3_2_vertical_velocity.png}
    \caption{presents the corresponding vertical velocity profile. The predicted 
    velocity captures the overall trend and curvature of the reference trajectory, 
    with noticeable divergence during the later phase of motion.}
    \label{fig:vertical_velocity}
\end{figure}


\subsection{Mass evolution}

\begin{figure}[htbp]
    \centering
    \includegraphics[width=14.5cm]{figures/evaluation_report/figures/figure_3_3_mass_evolution.png}
    \caption{compares the predicted mass evolution against the reference solution. During the 
    propulsive phase, the predicted mass follows the general decreasing trend of the reference. 
    After propellant depletion, the reference mass remains constant, while the predicted mass 
    exhibits small fluctuations around the terminal value.}
    \label{fig:mass_evolution}
\end{figure}

\subsection{Horizontal motion}

\begin{figure}[htbp]
    \centering
    \includegraphics[width=14.5cm]{figures/evaluation_report/figures/figure_3_4_horizontal_position.png}
    \caption{illustrates the horizontal position components. The reference trajectory remains at zero 
    in both horizontal directions, while the predicted trajectory exhibits small deviations, indicating 
    drift in lateral motion.}
    \label{fig:horizontal_position}
\end{figure}

\subsection{Attitude representation}

\begin{figure}[htbp]
    \centering
    \includegraphics[width=14.5cm]{figures/evaluation_report/figures/figure_3_5_quaternion_norm.png}
    \caption{reports the deviation of the predicted quaternion norm from unity over time. 
    The deviation remains extremely small throughout the trajectory, indicating that quaternion 
    normalization is largely preserved.}
    \label{fig:attitude_representation}
\end{figure}

\section{Quantitative Error Metrics}

This section evaluates prediction accuracy across the entire test set 
using quantitative error measures.

\subsection{Per-state RMSE statistics}

\begin{table}[htbp]
    \centering
    \begin{tabular}{lrrl}
\toprule
State Variable & Mean RMSE & Standard Deviation & Units \\
\midrule
x & 0.019133 & 0.000854 & nondim \\
y & 0.002535 & 0.001278 & nondim \\
z & 0.065454 & 0.015009 & nondim \\
vx & 0.275908 & 0.017068 & nondim \\
vy & 0.002568 & 0.001137 & nondim \\
vz & 0.575095 & 0.080124 & nondim \\
q0 & 0.072533 & 0.029386 & nondim \\
q1 & 0.000960 & 0.000707 & nondim \\
q2 & 0.335586 & 0.071237 & nondim \\
q3 & 0.000720 & 0.000347 & nondim \\
wx & 0.000801 & 0.000251 & nondim \\
wy & 0.008613 & 0.001579 & nondim \\
wz & 0.000704 & 0.000175 & nondim \\
m & 0.012871 & 0.007033 & nondim \\
\bottomrule
\end{tabular}

    \caption{summarizes the root-mean-square error for each state variable, 
    averaged across all test trajectories. Position, velocity, rotation, and mass 
    errors are reported separately to reflect their different physical scales.}
    \label{tab:per_state_rmse}
\end{table}

\subsection{Aggregate performance measures}

\begin{table}[htbp]
    \centering
    \begin{tabular}{llr}
\toprule
Metric & Observed Range & Test Set Size \\
\midrule
Vertical position accuracy (RMSE) & 0.067152 � 0.015009 & 20 \\
Velocity accuracy (RMSE) & 0.580650 � 0.080124 & 20 \\
Mass behavior (RMSE) & 0.014667 � 0.007033 & 20 \\
Quaternion norm range & [1.000000, 1.000000] & 20 \\
Residual magnitude range & [0.166489, 9.513170] & 20 \\
\bottomrule
\end{tabular}

    \caption{provides a condensed summary of overall error statistics, including mean and 
    dispersion measures across the test set. These metrics offer a compact overview of 
    prediction accuracy beyond individual state variables.}
    \label{tab:Aggregate_performance_measures}
\end{table}

\subsection{RMSE distribution}

\begin{figure}[htbp]
    \centering
    \includegraphics[width=14.5cm]{figures/evaluation_report/figures/figure_3_9_rmse_distribution.png}
    \caption{visualizes the distribution of RMSE values across all test trajectories for different 
    state groups. The logarithmic scale highlights variability and outliers in prediction performance.}
    \label{fig:RMSE_distribution}
\end{figure}

\section{Constraint Satisfaction Results}

This section examines whether the PINN predictions satisfy key physical constraints imposed during training.

\subsection{Physics residual norm}

\begin{figure}[htbp]
    \centering
    \includegraphics[width=14.5cm]{figures/evaluation_report/figures/figure_3_10_physics_residuals.png}
    \caption{shows the time evolution of the physics residual norm for the representative trajectory. 
    The residual magnitude varies over time and exhibits a notable decrease toward the end of the 
    trajectory.}
    \label{fig:Physics_residual_norm}
\end{figure}

\subsection{Mass monotonicity constraint}

\begin{figure}[htbp]
    \centering
    \includegraphics[width=14.5cm]{figures/evaluation_report/figures/figure_3_11_mass_monotonicity.png}
    \caption{evaluates compliance with the mass monotonicity constraint by plotting the time 
    derivative of mass. The reference trajectory shows no violations, while the predicted 
    trajectory exhibits multiple instances where the monotonicity condition is violated.}
    \label{fig:mass_monotonicity}
\end{figure}


\section{Summary of Observed Results}

The results presented in this chapter demonstrate that the proposed PINN is capable of 
reconstructing the overall structure of rocket trajectories for a representative test 
case, as shown by close visual agreement in vertical position and velocity. Quantitative 
evaluation across the test set indicates that prediction errors vary by state type, with 
mass and rotational states exhibiting distinct error distributions.

Constraint satisfaction analysis reveals that quaternion normalization is largely 
maintained, while mass monotonicity violations occur in the predicted trajectories. 
Physics residual magnitudes remain bounded and decrease toward the end of the simulated 
time horizon.

All observations reported in this chapter are based solely on empirical evaluation results 
without interpretation of underlying causes.